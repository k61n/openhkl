%--------------------------------------------------------------------------------------------------
\documentclass[twocolumn,preprintnumbers,amsmath,amssymb]{revtex4}
\usepackage[utf8]{inputenc}
\usepackage[english]{babel}
\usepackage{url}
\usepackage{amsmath}
\usepackage[nottoc,numbib]{tocbibind}


% convenience commands
\newcommand{\CChalf}{CC_{1/2}}
\newcommand{\CCtrue}{CC_{\mathrm{true}}}
\newcommand{\CCstar}{CC\ast}
\newcommand{\cov}{\mathrm{cov}}
\newcommand{\E}{\mathrm{E}}
\newcommand{\nsxtool}{{\tt{nsxtool}}}

\newcommand{\calP}{\mathcal{P}}
\newcommand{\calB}{\mathcal{B}}

\def\v#1{\mathbf{#1}}

% for placeholder text
% \usepackage{lipsum}

%--------------------------------------------------------------------------------------------------

\begin{document}

\title{NSXTool Internal Documentation: Notes on Literature}
\author{the NSXTool collaboration}

\date{internal notes, work in progress, do not circulate, \today}


\maketitle

\section{Kabsch 2010b}

Paper on ``Integration, scaling, space-group assignement and post-refinement'' \cite{Kab10b}.

\subsection{Notation correspondences}
%--------------------------------------------------------------------------------------------------

\begin{tabular}{lll}
&Kabsch &we\\\hline
incident wavevector &$\v{S}_0$ &$\v{k}_\text{i}$\\
reciprocal vector &$\v{p^*}$ &$\v{q}$\\
detector distance &$F$ &?\\
\end{tabular}

\subsection{Spot prediction (Sect 2.2)}
%--------------------------------------------------------------------------------------------------

This section uses vector components in the coordinate system $\{\v{m}_1,\v{m}_2,\v{m}_3\}$.
In this special system,
$\v{p}^*$ and $\v{p}^*_0$ are related to each other
through a rotation by $\varphi$ around $v{m}_2$,
and $\v{S}_0$ is perpendicular to $\v{m}_1$.

\subsubsection{How to obtain $\v{p}^*$ as function of $\v{p}^*_0,\varphi$}

In $\v{p}^*=D(\v{m}_2,\varphi)\v{p}^*_0$, just insert the standard rotation matrix~$D$.
Skip the line that involves the cross-product $\v{m}_2\times\v{p}^*_0$,
and directly obtain the next line.

\subsubsection{How to obtain $\v{p}^*$ as function of $\v{p}^*_0,\v{S}_0$}

First, the component along $\v{m}_2$: Rotation around $\v{m}_2$ does not change this component,
hence $\v{p}^*\v{m}_2 = \v{p}^*_0\v{m}_2$.

Next, the component along $\v{m}_3$:
By construction, $\v{m}_1\perp\v{S}_0$.
This simplifies the vector product
\begin{equation}
\begin{array}{ll}
   \v{S}_0\v{p}^*&=(\v{S_0}\v{m}_2)(\v{m}_2\v{p}^*) + (\v{S_0}\v{m}_3)(\v{m}_3\v{p}^*)
\\
                &=(\v{S_0}\v{m}_2)(\v{m}_2\v{p}^*_0) + (\v{S_0}\v{m}_3)(\v{m}_3\v{p}^*).
\end{array}
\end{equation}
Insert this in the Laue equation $-2\v{S}_0\v{p}^*={p^*_0}^2$,
and resolve for $\v{p}^*\v{m}_3$.

Finally, the component along $\v{m}_1$:
Resolve ${p^*_0}^2=\sum {(\v{p}^*\v{m}_i)}^2$ for $\v{p}^*\v{m}_1$.

\subsubsection{How to obtain $\varphi$ as function of $\v{p}^*_0,\v{p}^*$}

Start from
\begin{equation}
  \cos\varphi=\frac{{\v{p}^*}_\perp{\v{p}^*_0}_\perp}{|{\v{p}^*}_\perp|\cdot|{\v{p}^*_0}_\perp|}
\end{equation}
where the subscript~$\perp$ designates the projection into the plane
perpendicular to $\v{m_2}$.

%%%%%%%%%%%%%%%%%%%%%%%%%%%%%%%%%%%%%%%%%%%%%%%%%%%%%%%%%%%%%%%%%%%%%%%%%%%%%%%
\bibliographystyle{switch}
\bibliography{jw7}
%%%%%%%%%%%%%%%%%%%%%%%%%%%%%%%%%%%%%%%%%%%%%%%%%%%%%%%%%%%%%%%%%%%%%%%%%%%%%%%

\end{document}
